
\subsection{前言}

dfs(即 深搜)是一种常见的算法,大部分的题目都可以用 dfs 解决,但是大部分情况下,这都是骗分算法,很少会有爆搜为正解的题目。因为 dfs 的时间复杂度特别高。(没学过 dfs 的请自行补上这一课)

既然不能成为正解,那就多骗一点分吧。那么这一篇文章将介绍一些实用的优化算法(俗称 “剪枝”)。

先来一段深搜模板,之后的模板将在此基础上进行修改。

\begin{cppcode}
int ans = 最坏情况, now;  // now为当前答案
void dfs(传入数值) {
  if (到达目的地) ans = 从当前解与已有解中选最优;
  for (遍历所有可能性)
    if (可行) {
      进行操作;
      dfs(缩小规模);
      撤回操作;
    }
}
\end{cppcode}

其中的 ans 可以是解的记录,那么从当前解与已有解中选最优就变成了输出解.

\subsection{优化与剪枝}

最常用的剪枝有 3 种,记忆化搜索、最优性剪枝、可行性剪枝。

\subsubsection{记忆化搜索}

因为在搜索中,相同的传入值往往会带来相同的解,那我们就可以用数组来记忆,详见 记忆化搜索 。

\textbf{模板:}

\begin{cppcode}
int g[MAXN];  //定义记忆化数组
int ans = 最坏情况, now;
void dfs f(传入数值) {
  if (g[规模] != 无效数值) return;  //或记录解,视情况而定
  if (到达目的地) ans = 从当前解与已有解中选最优;  //输出解,视情况而定
  for (遍历所有可能性)
    if (可行) {
      进行操作;
      dfs(缩小规模);
      撤回操作;
    }
}
int main() {
  ... memset(g, 无效数值, sizeof(g));  //初始化记忆化数组
  ...
}
\end{cppcode}

\subsubsection{最优性剪枝}

在搜索中导致运行慢的原因还有一种,就是在当前解已经比已有解差时仍然在搜索,那么我们只需要判断一下当前解是否已经差于已有解。

\textbf{模板:}

\begin{cppcode}
int ans = 最坏情况, now;
void dfs(传入数值) {
  if (now比ans的答案还要差) return;
  if (到达目的地) ans = 从当前解与已有解中选最优;
  for (遍历所有可能性)
    if (可行) {
      进行操作;
      dfs(缩小规模);
      撤回操作;
    }
}
\end{cppcode}

\subsubsection{可行性剪枝}

\textbf{模板:}

在搜索中如果当前解已经不可用了还运行,也是在搜索中导致运行慢的原因。

\begin{cppcode}
int ans = 最坏情况, now;
void dfs(传入数值) {
  if (当前解已不可用) return;
  if (到达目的地) ans = 从当前解与已有解中选最优;
  for (遍历所有可能性)
    if (可行) {
      进行操作;
      dfs(缩小规模);
      撤回操作;
    }
}
\end{cppcode}
