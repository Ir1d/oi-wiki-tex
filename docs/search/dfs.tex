
DFS 全称是 \href{https://en.wikipedia.org/wiki/Depth-first_search}{Depth First Search}。

是一种图的遍历算法。

所谓深度优先。就是说每次都尝试向更深的节点走。

如果没有更深的节点了,就回到上一层的下一个节点继续刚才的过程。

上面的解释太过高深,我们可以感性地理解一下它,在较为初级的应用(非图论)中,搜索就是一个暴力枚举,

如这个例子:

\begin{QUOTE}{}{}
把正整数 n 分解为 3 个不同的数,如 6=1+2+3 排在后面的数必须大于等于前面的数



对于这个问题,如果不知道搜索,应该怎么办呢?



当然是 3 重循环 伪代码如下



\begin{minted}{text}

for i=1..n

  for j=i..n

    for k=1..n

      if (i+j+k=n) printf("%d=%d+%d+%d",n,i,j,k);

\end{minted}



那如果是分解成四个整数呢?



再加一重循环?
\end{QUOTE}

那分解成小于等于 m 个整数呢?

if 一大堆,写 m 个?

这时候就需要用到搜索了。

上面的例子也可以抽象成图,就是把能分解的数都算作一个点,然后按顺序把他们连起来。

\subsection{模板}

不管是图还是其它,都是这样

伪代码:

\begin{minted}{text}
dfs(n) {
  if (碰到边界) //如上面例子中的分解完就是基本情况
    返回 值,并退出
  for i=可以继续搜下去的情况
    if(可以){
      标记为不可以
      dfs(i);//继续往下搜
      标回可以
    }
}
\end{minted}

有些情况不需要标记,请自行判断。

\subsection{实现(对于图来说)}

伪代码:

\begin{minted}{text}
dfs(u) {
  visited[u] = true
  for each edge(u, v) {
    if (!visited[v]) {
      dfs(v)
    }
  }
}
\end{minted}

C++:

\begin{cppcode}
void dfs(int u) {
  vis[u] = 1;
  for (int i = head[u]; i; i = e[i].x) {
    // 这里用到的是链式前向星来存图
    if (!vis[e[i].t]) {
      dfs(v);
    }
  }
}
\end{cppcode}

时间复杂度 $O(n + m)$。

空间复杂度 $O(n)$。 (vis 数组和递归栈)

\subsection{在树 / 图上 DFS}

主条目: 在树 / 图上 DFS 
