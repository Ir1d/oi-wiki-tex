
在学习本章前请确认你已经学习了 动态规划部分简介 

树形 DP,即在树上进行的 DP。由于树固有的递归性质,树形 DP 一般都是递归进行的。

\subsection{例题}

以下面这道题为例,介绍一下树形 DP 的一般过程。

\begin{NOTE}{例题 \href{https://www.luogu.org/problemnew/show/P1352}{洛谷 P1352 没有上司的舞会}}{}
某大学有 $n$ 个职员,编号为 $1\text{~} N$ 。他们之间有从属关系,也就是说他们的关系就像一棵以校长为根的树,父结点就是子结点的直接上司。现在有个周年庆宴会,宴会每邀请来一个职员都会增加一定的快乐指数 $a_i$,但是呢,如果某个职员的上司来参加舞会了,那么这个职员就无论如何也不肯来参加舞会了。所以,请你编程计算,邀请哪些职员可以使快乐指数最大,求最大的快乐指数。

\end{NOTE}


我们可以定义 $f(i,0/1)$ 代表以 $i$ 为根的子树的最优解(第二维的值为 0 代表 $i$ 不参加舞会的情况,1 代表 $i$ 参加舞会的情况)。

显然,我们可以推出下面两个状态转移方程(其中下面的 $x$ 都是 $i$ 的儿子):

\begin{itemize}
\item $f(i,0) = \sum\max \{f(x,1),f(x,0)\}$ (上司不参加舞会时,下属可以参加,也可以不参加)
\item $f(i,1) = \sum{f(x,0)} + a_i$ (上司参加舞会时,下属都不会参加)
\end{itemize}

我们可以通过 DFS,在返回上一层时更新当前节点的最优解。

代码:

\begin{cppcode}
#include <algorithm>
#include <cstdio>
using namespace std;
struct edge {
  int v, next;
} e[6005];
int head[6005], n, cnt, f[6005][2], ans, is_h[6005], vis[6005];
void addedge(int u, int v) {
  e[++cnt].v = v;
  e[cnt].next = head[u];
  head[u] = cnt;
}
void calc(int k) {
  vis[k] = 1;
  for (int i = head[k]; i; i = e[i].next)  //枚举该节点的每个子节点
  {
    if (vis[e[i].v]) continue;
    calc(e[i].v);
    f[k][1] += f[e[i].v][0];
    f[k][0] += max(f[e[i].v][0], f[e[i].v][1]);
  }
  return;
}
int main() {
  scanf("%d", &n);
  for (int i = 1; i <= n; i++) scanf("%d", &f[i][1]);
  for (int i = 1; i < n; i++) {
    int l, k;
    scanf("%d%d", &l, &k);
    is_h[l] = 1;
    addedge(k, l);
  }
  for (int i = 1; i <= n; i++)
    if (!is_h[i])  //从根节点开始DFS
    {
      calc(i);
      printf("%d", max(f[i][1], f[i][0]));
      return 0;
    }
}
\end{cppcode}

\subsection{习题}

\href{http://acm.hdu.edu.cn/showproblem.php?pid=2196}{HDU 2196 Computer}

\href{http://poj.org/problem?id=1463}{POJ 1463 Strategic game}

\href{http://poj.org/problem?id=3585}{POJ 3585 Accumulation Degree}
