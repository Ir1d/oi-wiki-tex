
\subsection{经典题型}

数位 DP 问题往往都是这样的题型,给定一个闭区间 $[l,r]$,让你求这个区间中满足 \textbf{某种条件} 的数的总数。

\begin{NOTE}{例题 \href{https://www.luogu.org/problemnew/show/P2657}{洛谷 P2657 windy 数}}{}
题目大意:给定一个区间 $[l,r]$ ,求其中满足条件 \textbf{不含前导 $0$ 且相邻两个数字相差至少为 $2$} 的数字个数。

\end{NOTE}


首先我们将问题转化成更加简单的形式。设 $ans_i$ 表示在区间 $[1,i]$ 中满足条件的数的数量,那么所求的答案就是 $ans_r-ans_{l-1}$。

分开求解这两个问题。

对于一个小于 $n$ 的数,它从高到低肯定出现某一位,使得这一位上的数值小于 $n$ 这一位上对应的数值。而之前的所有位都和 $n$ 上的位相等。

有了这个性质,我们可以定义 $f(i,st,op)$ 表示当前将要考虑的是从高到低的第 $i$ 位,当前该前缀的状态为 $st$ 且前缀和当前求解的数字的大小关系是 $op$ ($op=1$ 表示等于,$op=0$ 表示小于)时的数字个数。在本题中,这个前缀的状态就是上一位的值,因为当前将要确定的位不能取哪些数只和上一位有关。在其他题目中,这个值可以是:前缀的数字和,前缀所有数字的 $\gcd$,该前缀取模某个数的余数,也有两种或多种合用的情况。

写出 \textbf{状态转移方程} : $f(i,st,op)=\sum_{i=1}^{maxx} f(i+1,k,op=1~ \operatorname{and}~ i=maxx )\quad (|st-k|\ge 2)$

这里的 $k$ 就是当前枚举的下一位的值,而 $maxx$ 就是当前能取到的最高位。因为如果 $op=1$,那么你在这一位上取的值一定不能大于求解的数字上该位的值,否则则没有限制。

我们发现,尽管前缀所选择的状态不同,而 $f$ 的三个参数相同,答案就是一样的。为了防止这个答案被计算多次,可以使用记忆化搜索的方式实现。

核心代码:

\begin{cppcode}
int dfs(int x, int st, int op)  // op=1 =;op=0 <
{
  if (!x) return 1;
  if (!op && ~f[x][st]) return f[x][st];
  int maxx = op ? dim[x] : 9, ret = 0;
  for (int i = 0; i <= maxx; i++) {
    if (abs(st - i) < 2) continue;
    if (st == 11 && i == 0)
      ret += dfs(x - 1, 11, op & (i == maxx));
    else
      ret += dfs(x - 1, i, op & (i == maxx));
  }
  if (!op) f[x][st] = ret;
  return ret;
}
int solve(int x) {
  memset(f, -1, sizeof f);
  dim.clear();
  dim.push_back(-1);
  int t = x;
  while (x) {
    dim.push_back(x % 10);
    x /= 10;
  }
  return dfs(dim.size() - 1, 11, 1);
}
\end{cppcode}

\subsection{几道练习题}

\href{https://www.lydsy.com/JudgeOnline/problem.php?id=3679}{BZOJ 3679 数字之积 }

\href{https://www.luogu.org/problemnew/show/P2602}{洛谷 P2602 数字计数 }

\href{https://www.luogu.org/problemnew/show/P4127}{洛谷 P4127 同类分布 }

\href{https://www.luogu.org/problemnew/show/P3413}{洛谷  P3413 SAC - 萌数}

\href{http://acm.hdu.edu.cn/showproblem.php?pid=6148}{HDU 6148 Valley Number }

\href{http://codeforces.com/problemset/problem/55/D}{CF55D Beautiful numbers}

\href{http://codeforces.com/problemset/problem/628/D}{CF628D Magic Numbers}

\href{http://codeforces.com/problemset/problem/401/D}{CF401D Roman and Numbers}
