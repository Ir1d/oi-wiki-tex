
\subsection{描述}

给定一个长度为 $n$ 的字符串 $s$,请找到所有对 $(i, j)$ 使得子串 $s[i \dots j]$ 为一个回文串。当 $t = t_{\text{rev}}$ 时,字符串 $t$ 是一个回文串($t_{\text{rev}}$ 是 $t$ 的反转字符串)。

\subsection{更进一步的描述}

显然在最坏情况下可能有 $O(n^2)$ 个回文串,因此似乎一眼看过去该问题并没有线性算法。

但是关于回文串的信息可用\textbf{一种更紧凑的方式}表达:对于每个位置 $i = 0 \dots n - 1$,我们找出值 $d_1[i]$ 和 $d_2[i]$。二者分别表示以位置 $i$ 为中心的长度为奇数和长度为偶数的回文串个数。

举例来说,字符串 $s = \mathtt{abababc}$ 以 $s[3] = b$ 为中心有三个奇数长度的回文串,也即 $d_1[3] = 3$:

$$
a\ \overbrace{b\ a\ \underset{s_3}{b}\ a\ b}^{d_1[3]=3}\ c
$$

字符串 $s = \mathtt{cbaabd}$ 以 $s[3] = a$ 为中心有两个偶数长度的回文串,也即 $d_2[3] = 2$:

$$
c\ \overbrace{b\ a\ \underset{s_3}{a}\ b}^{d_2[3]=2}\ d
$$

因此关键思路是,如果以某个位置 $i$ 为中心,我们有一个长度为 $l$ 的回文串,那么我们有以 $i$ 为中心的长度为 $l - 2$,$l - 4$,等等的回文串。所以 $d_1[i]$ 和 $d_2[i]$ 两个数组已经足够表示字符串中所有子回文串的信息。

一个令人惊讶的事实是,存在一个复杂度为线性并且足够简单的算法计算上述两个 “回文性质数组” $d_1[]$ 和 $d_2[]$。在这篇文章中我们将详细的描述该算法。

\subsection{解法}

总的来说,该问题具有多种解法:应用字符串哈希,该问题可在 $O(n \log n)$ 时间内解决,而使用后缀数组和快速 LCA 该问题可在 $O(n)$ 时间内解决。

但是这里描述的算法\textbf{压倒性}的简单,并且在时间和空间复杂度上具有更小的常数。该算法由\textbf{Glenn K. Manacher}在 1975 年提出。

\subsection{朴素算法}

为了避免在之后的叙述中出现歧义,这里我们指出什么是 “朴素算法”。

该算法通过下述方式工作:对每个中心位置 $i$,在比较一对对应字符后,只要可能,该算法便尝试将答案加 $1$。

该算法是比较慢的:它只能在 $O(n^2)$ 的时间内计算答案。

该朴素算法的实现如下:

\begin{cppcode}
vector<int> d1(n), d2(n);
for (int i = 0; i < n; i++) {
  d1[i] = 1;
  while (0 <= i - d1[i] && i + d1[i] < n && s[i - d1[i]] == s[i + d1[i]]) {
    d1[i]++;
  }

  d2[i] = 0;
  while (0 <= i - d2[i] - 1 && i + d2[i] < n &&
         s[i - d2[i] - 1] == s[i + d2[i]]) {
    d2[i]++;
  }
}
\end{cppcode}

\subsection{Manacher 算法}

这里我们将只描述算法中寻找所有奇数长度子回文串的情况,即只计算 $d_1[]$;寻找所有偶数长度子回文串的算法(即计算数组 $d_2[]$)将只需对奇数情况下的算法进行一些小修改。

为了快速计算,我们维护已找到的子回文串的最靠右的\textbf{边界 $(l, r)$}(即具有最大 $r$ 值的回文串)。初始时,我们置 $l = 0$ 和 $r = -1$。

现在假设我们要对下一个 $i$ 计算 $d_1[i]$,而之前所有 $d_1[]$ 中的值已计算完毕。我们将通过下列方式计算:

\begin{itemize}
\item 如果 $i$ 位于当前子回文串之外,即 $i > r$,那么我们调用朴素算法。
因此我们将连续的增加 $d_1[i]$,同时在每一步中检查当前的子串 $[i - d_1[i] \dots i + d_1[i]]$ 是否为一个回文串。如果我们找到了第一处对应字符不同,又或者碰到了 $s$ 的边界,则算法停止。在两种情况下我们均已计算完 $d_1[i]$。此后,仍需记得更新 $(l, r)$。
\item 现在考虑 $i \le r$ 的情况。我们将尝试从已计算过的 $d_1[]$ 的值中获取一些信息。首先在子回文串 $(l, r)$ 中反转位置 $i$,即我们得到 $j = l + (r - i)$。现在来考察值 $d_1[j]$。因为位置 $j$ 同位置 $i$ 对称,我们\textbf{几乎总是}可以置 $d_1[i] = d_1[j]$。该想法的图示如下(可认为以 $j$ 为中心的回文串被 “拷贝” 至以 $i$ 为中心的位置上):
$$
\ldots\ 
\overbrace{
    s_l\ \ldots\ 
    \underbrace{
        s_{j-d_1[j]+1}\ \ldots\ s_j\ \ldots\ s_{j+d_1[j]-1}
    }_\text{palindrome}\ 
    \ldots\ 
    \underbrace{
        s_{i-d_1[j]+1}\ \ldots\ s_i\ \ldots\ s_{i+d_1[j]-1}
    }_\text{palindrome}\ 
    \ldots\ s_r
}^\text{palindrome}\ 
\ldots
$$
然而有一个\textbf{棘手的情况}需要被正确处理:当 “内部” 的回文串到达 “外部” 回文串的边界时,即 $j - d_1[j] + 1 \le l$(或者等价的说,$i + d_1[j] - 1 \ge r$)。因为在 “外部” 回文串范围以外的对称性没有保证,因此直接置 $d_1[i] = d_1[j]$ 将是不正确的:我们没有足够的信息来断言在位置 $i$ 的回文串具有同样的长度。
实际上,为了正确处理这种情况,我们应该 “截断” 回文串的长度,即置 $d_1[i] = r - i$。之后我们将运行朴素算法以尝试尽可能增加 $d_1[i]$ 的值。
该种情况的图示如下(以 $j$ 为中心的回文串已经被截断以落在 “外部” 回文串内):
$$
\ldots\ 
\overbrace{
    \underbrace{
        s_l\ \ldots\ s_j\ \ldots\ s_{j+(j-l)}
    }_\text{palindrome}\ 
    \ldots\ 
    \underbrace{
        s_{i-(r-i)}\ \ldots\ s_i\ \ldots\ s_r
    }_\text{palindrome}
}^\text{palindrome}\ 
\underbrace{
    \ldots \ldots \ldots \ldots \ldots
}_\text{try moving here}
$$
该图示显示出,尽管以 $j$ 为中心的回文串可能更长,以致于超出 “外部” 回文串,但在位置 $i$,我们只能利用其完全落在 “外部” 回文串内的部分。然而位置 $i$ 的答案可能比这个值更大,因此接下来我们将运行朴素算法来尝试将其扩展至 “外部” 回文串之外,也即标识为 "try moving here" 的区域。
\end{itemize}

最后,仍有必要提醒的是,我们应当记得在计算完每个 $d_1[i]$ 后更新值 $(l, r)$。

同时,再让我们重复一遍:计算偶数长度回文串数组 $d_2[]$ 的算法同上述计算奇数长度回文串数组 $d_1[]$ 的算法十分类似。

\subsection{Manacher 算法的复杂度}

因为在计算一个特定位置的答案时我们总会运行朴素算法,所以一眼看去该算法的时间复杂度为线性的事实并不显然。

然而更仔细的分析显示出该算法具有线性复杂度。此处我们需要指出, 计算 Z 函数的算法 和该算法较为类似,并同样具有线性时间复杂度。

实际上,注意到朴素算法的每次迭代均会使 $r$ 增加 $1$,以及 $r$ 在算法运行过程中从不减小。这两个观察告诉我们朴素算法总共会进行 $O(n)$ 次迭代。

Manacher 算法的另一部分显然也是线性的,因此总复杂度为 $O(n)$。

\subsection{Manacher 算法的实现}

\subsubsection{分类讨论}

为了计算 $d_1[]$,我们有以下代码:

\begin{cppcode}
vector<int> d1(n);
for (int i = 0, l = 0, r = -1; i < n; i++) {
  int k = (i > r) ? 1 : min(d1[l + r - i], r - i);
  while (0 <= i - k && i + k < n && s[i - k] == s[i + k]) {
    k++;
  }
  d1[i] = k--;
  if (i + k > r) {
    l = i - k;
    r = i + k;
  }
}
\end{cppcode}

计算 $d_2[]$ 的代码十分类似,但是在算术表达式上有些许不同:

\begin{cppcode}
vector<int> d2(n);
for (int i = 0, l = 0, r = -1; i < n; i++) {
  int k = (i > r) ? 0 : min(d2[l + r - i + 1], r - i + 1);
  while (0 <= i - k - 1 && i + k < n && s[i - k - 1] == s[i + k]) {
    k++;
  }
  d2[i] = k--;
  if (i + k > r) {
    l = i - k - 1;
    r = i + k;
  }
}
\end{cppcode}

\subsubsection{统一处理}

虽然在讲解过程及上述实现中我们将 $d_1[]$ 和 $d_2[]$ 的计算分开考虑,但实际上可以通过一个技巧将二者的计算统一为 $d_1[]$ 的计算。

给定一个长度为 $n$ 的字符串 $s$,我们在其 $n + 1$ 个空中插入分隔符 $\#$,从而构造一个长度为 $2n + 1$ 的字符串 $s'$。举例来说,对于字符串 $s = \mathtt{abababc}$,其对应的 $s' = \mathtt{\#a\#b\#a\#b\#a\#b\#c\#}$。

对于字母间的 $\#$,其实际意义为 $s$ 中对应的 “空”。而两端的 $\#$ 则是为了实现的方便。

注意到,在对 $s'$ 计算 $d_1[]$ 后,对于一个位置 $i$,$d_1[i]$ 所描述的最长的子回文串必定以 $\#$ 结尾(若以字母结尾,由于字母两侧必定各有一个 $\#$,因此可向外扩展一个得到一个更长的)。因此,对于 $s$ 中一个以字母为中心的极大子回文串,设其长度为 $m + 1$,则其在 $s'$ 中对应一个以相应字母为中心,长度为 $2m + 3$ 的极大子回文串;而对于 $s$ 中一个以空为中心的极大子回文串,设其长度为 $m$,则其在 $s'$ 中对应一个以相应表示空的 $\#$ 为中心,长度为 $2m + 1$ 的极大子回文串(上述两种情况下的 $m$ 均为偶数,但该性质成立与否并不影响结论)。综合以上观察及少许计算后易得,在 $s'$ 中,$d_1[i]$ 表示在 $s​$ 中以对应位置为中心的极大子回文串的\textbf{总长度加一}。

上述结论建立了 $s'$ 的 $d_1[]$ 同 $s$ 的 $d_1[]$ 和 $d_2[]$ 间的关系。

由于该统一处理本质上即求 $s'$ 的 $d_1[]$,因此在得到 $s'$ 后,代码同上节计算 $d_1[]$ 的一样。

\subsection{练习题目}

\begin{itemize}
\item \href{https://uva.onlinejudge.org/index.php?option=com_onlinejudge&Itemid=8&page=show_problem&problem=2470}{UVA \textbackslash{}\#11475 "Extend to Palindrome"}
\item \href{https://www.luogu.org/problemnew/show/P4555}{P4555 国家集训队 最长双回文串}
\end{itemize}

\hr

\textbf{本页面主要译自博文 \href{http://e-maxx.ru/algo/palindromes_count}{Нахождение всех подпалиндромов} 与其英文翻译版 \href{https://cp-algorithms.com/string/manacher.html}{Finding all sub-palindromes in \$O(N)\$} 。其中俄文版版权协议为 Public Domain + Leave a Link;英文版版权协议为 CC-BY-SA 4.0。}
