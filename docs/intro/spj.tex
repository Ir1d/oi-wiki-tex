
一道题如果有多组解,我们就需要一个程序来判断答案合法性,这便是 Special Judge (spj),又常被称作 checker,下面介绍部分评测工具 /OJ 的 spj 编写方法。

\begin{WARNING}{}{}
spj 还应当判断文件尾是否有多余内容,及输出格式是否正确(如题目要求数字间用一个空格隔开,而选手却使用了换行)。但是,目前前者只有 Testlib 可以方便地做到这一点,而后者几乎无人去特意进行这种判断。

浮点数时应注意 nan,不合理的判断方式会导致输出 nan 即可 AC。

\end{WARNING}


以下均使用 C++,以 “要求标准答案与选手答案差值小于 1e-3,文件名为 num” 为例。

\subsection{Testlib}

\href{https://codeforces.com/testlib}{Testlib} 是一个强大的算法竞赛题目辅助系统,只需要在程序中引入 \href{https://github.com/MikeMirzayanov/testlib/blob/master/testlib.h}{testlib.h} 头文件即可使用。用法详见  Testlib  页面。

使用 Testlib 编写 spj 的好处为我们不再需要判断文件尾的多余内容,其会帮助我们自动判断,也无需担忧 nan。

必须使用 Testlib 做 spj 的 评测工具 /OJ:Codeforces、洛谷、UOJ 等

可以使用 Testlib 做 spj 的 评测工具 /OJ:LibreOJ(SYZOJ 2)、Lemon 等

SYZOJ 2 所需的修改版 Testlib 可以在\href{https://pastebin.com/3GANXMG7}{这里}获取到,感谢 \href{https://loj.ac/article/124}{cyand1317}。

Lemon 所需的修改版 Testlib 可以在\href{https://paste.ubuntu.com/p/JsTspHHnmB/}{这里}获取到,感谢 matthew99。注意此版本 Testlib 注册 checker 应使用 \texttt{registerLemonChecker()} 而非 \texttt{registerTestlibCmd()}。

其他评测工具 /OJ 大部分需要按照其 spj 编写格式修改 Testlib。

\begin{cppcode}
#include <testlib.h>
#include <cmath>
int main(int argc, char *argv[]) {
  /*
   * inf:输入
   * ouf:选手输出
   * ans:标准输出
   */
  registerTestlibCmd(argc, argv);
  double pans = ouf.readDouble(), jans = ans.readDouble();
  if (abs(pans - jans) < 1e-3)
    quitf(_ok, "Good job");
  else
    quitf(_wa, "Too big or too small, expected %f, found %f", jans, pans);
}
\end{cppcode}

\subsection{Lemon}

\textbf{Lemon 有现成的修改版 Testlib,建议使用 Testlib,见  Testlib }

\begin{cppcode}
#include <cmath>
#include <cstdio>
int main(int argc, char* argv[]) {
  /*
   * argv[1]:输入
   * argv[2]:选手输出
   * argv[3]:标准输出
   * argv[4]:单个测试点分值
   * argv[5]:输出最终得分
   * argv[6]:输出错误报告
   */
  FILE* fin = fopen(argv[1], "r");
  FILE* fout = fopen(argv[2], "r");
  FILE* fstd = fopen(argv[3], "r");
  FILE* fscore = fopen(argv[5], "w");
  FILE* freport = fopen(argv[6], "w");
  double pans, jans;
  fscanf(fout, "%lf", &pans);
  fscanf(fstd, "%lf", &jans);
  if (abs(pans - jans) < 1e-3) {
    fprintf(fscore, "%s", argv[4]);
    fprintf(freport, "Good job");
  } else {
    fprintf(fscore, "%d", 0);
    fprintf(freport, "Too big or too small, expected %f, found %f", jans, pans);
  }
}
\end{cppcode}

\subsection{Cena}

\begin{cppcode}
#include <cmath>
#include <cstdio>
int main(int argc, char* argv[]) {
  /*
   * FILENAME.in:输入
   * FILENAME.out:选手输出
   * argv[2]:标准输出
   * argv[1]:单个测试点分值
   * score.log:输出最终得分
   * report.log:输出错误报告
   */
  FILE* fin = fopen("num.in", "r");
  FILE* fout = fopen("num.out", "r");
  FILE* fstd = fopen(argv[2], "r");
  FILE* fscore = fopen("score.log", "w");
  FILE* freport = fopen("report.log", "w");
  double pans, jans;
  fscanf(fout, "%lf", &pans);
  fscanf(fstd, "%lf", &jans);
  if (abs(pans - jans) < 1e-3) {
    fprintf(fscore, "%s", argv[1]);
    fprintf(freport, "Good job");
  } else {
    fprintf(fscore, "%d", 0);
    fprintf(freport, "Too big or too small, expected %f, found %f", jans, pans);
  }
}
\end{cppcode}

\subsection{CCR}

\begin{cppcode}
#include <cmath>
#include <cstdio>
int main(int argc, char* argv[]) {
  /*
   * stdin:输入
   * argv[3]:选手输出
   * argv[2]:标准输出
   * stdout:L1:输出最终得分比率
   * stdout:L2:输出错误报告
   */
  FILE* fout = fopen(argv[3], "r");
  FILE* fstd = fopen(argv[2], "r");
  double pans, jans;
  fscanf(fout, "%lf", &pans);
  fscanf(fstd, "%lf", &jans);
  if (abs(pans - jans) < 1e-3) {
    printf("%d\n", 1);
    printf("Good job");
  } else {
    printf("%d\n", 0);
    printf("Too big or too small, expected %f, found %f", jans, pans);
  }
}
\end{cppcode}

\subsection{Arbiter}

\subsection{HUSTOJ}

\begin{cppcode}
#include <cmath>
#include <cstdio>
#define AC 0
#define WA 1
int main(int argc, char* argv[]) {
  /*
   * argv[1]:输入
   * argv[3]:选手输出
   * argv[2]:标准输出
   * exit code:返回判断结果
   */
  FILE* fout = fopen(argv[3], "r");
  FILE* fstd = fopen(argv[2], "r");
  double pans, jans;
  fscanf(fout, "%lf", &pans);
  fscanf(fstd, "%lf", &jans);
  if (abs(pans - jans) < 1e-3)
    return AC;
  else
    return WA;
}
\end{cppcode}

\subsection{QDUOJ}

QDUOJ 就麻烦一点,因为它的带 spj 的题目没有标准输出,只能把 std 写进 spj 跑出标准输出再判断。

\begin{cppcode}
#include <cmath>
#include <cstdio>
#define AC 0
#define WA 1
#define ERROR -1
double solve(...) {
  // std
}
int main(int argc, char* argv[]) {
  /*
   * argv[1]:输入
   * argv[2]:选手输出
   * exit code:返回判断结果
   */
  FILE* fin = fopen(argv[1], "r");
  FILE* fout = fopen(argv[2], "r");
  //读入
  double pans, jans;
  fscanf(fout, "%lf", &pans);
  jans = solve(...);
  if (abs(pans - jans) < 1e-3)
    return AC;
  else
    return WA;
}
\end{cppcode}

\subsection{LibreOJ(SYZOJ 2)}

\textbf{LibreOJ(SYZOJ 2) 有现成的修改版 Testlib,建议使用 Testlib,见  Testlib }

\begin{cppcode}
#include <cmath>
#include <cstdio>
int main(int argc, char* argv[]) {
  /*
   * in:输入
   * user_out:选手输出
   * answer:标准输出
   * code:选手代码
   * stdout:输出最终得分
   * stderr:输出错误报告
   */
  FILE* fin = fopen("in", "r");
  FILE* fout = fopen("user_out", "r");
  FILE* fstd = fopen("answer", "r");
  FILE* fcode = fopen("code", "r");
  double pans, jans;
  fscanf(fout, "%lf", &pans);
  fscanf(fstd, "%lf", &jans);
  if (abs(pans - jans) < 1e-3) {
    printf("%d", 100);
    fprintf(stderr, "Good job");
  } else {
    printf("%d", 0);
    fprintf(stderr, "Too big or too small, expected %f, found %f", jans, pans);
  }
}
\end{cppcode}
