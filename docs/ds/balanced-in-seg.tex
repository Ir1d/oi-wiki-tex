
\section{线段树套平衡树}

\subsubsection{常见用途}

在算法竞赛中,我们有时需要维护多维度信息。在这种时候,我们经常需要树套树来记录信息。当需要维护前驱,后继,第 $k$ 大,某个数的排名,或者插入删除的时候,我们通常需要使用平衡树来满足我们的需求,即线段树套平衡树。

\subsubsection{实现原理}

我们以\textbf{二逼平衡树}为例,来解释实现原理。

关于树套树的构建,我们对于外层线段树正常建树,对于线段树上的某一个节点,建立一棵平衡树,包含该节点所覆盖的序列。具体操作时我们可以将序列元素一个个插入,每经过一个线段树节点,就将该元素加入到该节点的平衡树中。

操作一,求某区间中某值的排名:我们对于外层线段树正常操作,对于在某区间中的节点的平衡树,我们返回平衡树中比该值小的元素个数,合并区间时,我们将小的元素个数求和即可。最后将返回值 $+1$,即为某值在某区间中的排名。

操作二,求某区间中排名为 $k$ 的值:我们可以采用二分策略。因为一个元素可能存在多个,其排名为一区间,且有些元素原序列不存在。所以我们采取和操作一类似的思路,我们用小于该值的元素个数作为参考进行二分,即可得解。

操作三,将某个数替换为另外一个数:我们只要在所有包含某数的平衡树中删除某数,然后再插入另外一个数即可。外层依旧正常线段树操作。

操作四,求某区间中某值的前驱:我们对于外层线段树正常操作,对于在某区间中的节点的平衡树,我们返回某值在该平衡树中的前驱,线段树的区间结果合并时,我们取最大值即可。

\subsubsection{空间复杂度}

我们每个元素加入 $\log n$ 个平衡树,所以空间复杂度为 $(n + q)\log{n}$。

\subsubsection{时间复杂度}

对于 $1,2,4$ 操作,我们考虑我们在外层线段树上进行 $\log{n}$ 次操作,每次操作会在一个内层平衡树树上进行 $\log{n}$ 次操作,所以时间复杂度为 $\log^2{n}$。

对于 $3$ 操作,多一个二分过程,为 $\log^3{n}$。

\subsubsection{经典例题}

\href{https://www.lydsy.com/JudgeOnline/problem.php?id=3196}{二逼平衡树} 外层线段树,内层平衡树。

\subsubsection{示例代码}

平衡树部分代码请参考 Splay 等其他条目。 传送至 Splay 条目 

操作一

\begin{cppcode}
int vec_rank(int k, int l, int r, int x, int y, int t) {
  if (x <= l && r <= y) {
    return spy[k].chk_rank(t);
  }
  int mid = l + r >> 1;
  int res = 0;
  if (x <= mid) res += vec_rank(k << 1, l, mid, x, y, t);
  if (y > mid) res += vec_rank(k << 1 | 1, mid + 1, r, x, y, t);
  if (x <= mid && y > mid) res--;
  return res;
}
\end{cppcode}

操作二

\begin{cppcode}
int el = 0, er = 100000001, emid;
while (el != er) {
  emid = el + er >> 1;
  if (vec_rank(1, 1, n, tl, tr, emid) - 1 < tk)
    el = emid + 1;
  else
    er = emid;
}
printf("%d\n", el - 1);
\end{cppcode}

操作三

\begin{cppcode}
void vec_chg(int k, int l, int r, int loc, int x) {
  int t = spy[k].find(dat[loc]);
  spy[k].dele(t);
  spy[k].insert(x);
  if (l == r) return;
  int mid = l + r >> 1;
  if (loc <= mid) vec_chg(k << 1, l, mid, loc, x);
  if (loc > mid) vec_chg(k << 1 | 1, mid + 1, r, loc, x);
}
\end{cppcode}

操作四

\begin{cppcode}
int vec_front(int k, int l, int r, int x, int y, int t) {
  if (x <= l && r <= y) return spy[k].chk_front(t);
  int mid = l + r >> 1;
  int res = 0;
  if (x <= mid) res = max(res, vec_front(k << 1, l, mid, x, y, t));
  if (y > mid) res = max(res, vec_front(k << 1 | 1, mid + 1, r, x, y, t));
  return res;
}
\end{cppcode}

\subsubsection{相关算法}

面对多维度信息的题目时,如果题目没有要求强制在线,我们还可以考虑 \textbf{CDQ 分治},或者\textbf{整体二分}等分治算法,来避免使用高级数据结构,减少代码实现难度。
