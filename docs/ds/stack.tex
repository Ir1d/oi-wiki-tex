
\subsection{栈}

栈是 OI 中常用的一种线性数据结构,请注意,本文主要讲的是栈这种数据结构, 而非程序运行时的系统栈 / 栈空间

栈的修改是按照后进先出的原则进行的,因此栈通常被称为是后进先出(last in first out)表,简称 LIFO 表。

\begin{NOTE}{warning}{}
为什么不是 FILO 呢?

\end{NOTE}


我们可以方便的使用数组来模拟一个栈, 如下 :

\begin{cppcode}
int stk[N];
// 这里使用 stk[0]( 即 *stk ) 代表栈中元素数量,同时也是栈顶下标
// 压栈 :
stk[++*stk] = var1;
// 取栈顶 :
int u = stk[*stk];
// 弹栈 :注意越界问题, *stk == 0 时不能继续弹出
if (*stk) --*stk;
// 清空栈
*stk = 0;
\end{cppcode}

同时 STL 也提供了一个方法 \texttt{std :: stack}

\begin{cppcode}
#include <stack>
// stack 构造 :
1. stack<Typename T> s;
2. stack<Typename T, Container> s;
/* stack 的 Container 需要满足有如下接口 :
 * back()
 * push_back()
 * pop_back()
 * 标准容器 std :: vector / deque / list 满足这些要求
 * 如使用 1 方式构造,默认容器使用 deque
 */
\end{cppcode}

\texttt{std :: stack} 支持赋值运算符 \texttt{=}

元素访问 :

\texttt{s.top()} 返回栈顶

容量 :

\texttt{s.empty()} 返回是否为空

\texttt{s.size()} 返回元素数量

修改 :

\texttt{s.push()} 插入传入的参数到栈顶

\texttt{s.pop()} 弹出栈顶

其他运算符 :

\texttt{==}、\texttt{!=}、\texttt{<}、\texttt{<=}、\texttt{>}、\texttt{>=} 可以按照字典序比较两个 \texttt{stack} 的值
