
在学习单调队列前,让我们先来看一道例题

\subsection{例题}

\href{http://poj.org/problem?id=2823}{Sliding Window}

本题大意是给出一个长度为 $n$ 的数组,编程输出每 $k$ 个连续的数中的最大值和最小值

最常用(\sout{暴力})的想法很简单,对于每一段 $i \sim i+k-1$ 的序列,逐个比较来找出最大值(和最小值),时间复杂度约为 $O(n \times k)$ 。

很显然,这其中进行了大量重复工作,除了开头 $k-1$ 个和结尾 $k-1$ 个数之外,每个数都进行了 $k$ 次比较,而题中 $100\%$ 的数据为 $n \le 1000000$ ,当 $k$ 稍大的情况下,显然会出现 TLE

这时所用到的就是单调队列了

\subsection{概念}

顾名思义,单调队列的重点分为 "单调" 和 "队列"

"单调" 指的是元素的的 "规律"——递增 (或递减)

"队列" 指的是元素只能从队头和队尾进行操作

Ps. 单调队列中的 "队列" 与正常的队列有一定的区别,稍后会提到

\subsection{例题分析}

有了上面 "单调队列" 的概念,很容易想到用单调队列进行优化

要求的是每连续的 $k$ 个数中的最大(最小)值,很明显,当一个数进入所要 "寻找" 最大值的范围中时,若这个数比其前面(先进队)的数要大,显然,前面的数会比这个数先出队且不再可能是最大值

也就是说——当满足以上条件时,可将前面的数 "弹出",再将该数真正 push 进队尾

这就相当于维护了一个递减的队列,符合单调队列的定义,减少了重复的比较次数,不仅如此,由于维护出的队伍是查询范围内的且是递减的,队头必定是该查询区域内的最大值,因此输出时只需输出队头即可

显而易见的是,在这样的算法中,每个数只要进队与出队各一次,因此时间复杂度被降到了 $O(N)$

而由于查询区间长度是固定的,超出查询空间的值再大也不能输出,因此还需要 site 数组记录第 $i$ 个队中的数在原数组中的位置,以弹出越界的队头

\href{https://www.luogu.org/paste/dze1lw3b}{例题代码}

Ps. 此处的 "队列" 跟普通队列的一大不同就在于可以从队尾进行操作, C++ 中有相似的数据结构 deque
