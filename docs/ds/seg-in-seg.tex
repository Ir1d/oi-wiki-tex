
\section{线段树套线段树}

\subsubsection{常见用途}

在算法竞赛中,我们有时需要维护多维度信息。在这种时候,我们经常需要树套树来记录信息。

\subsubsection{实现原理}

我们考虑用树套树如何实现在二维平面上进行单点修改,区域查询。我们考虑外层的线段树,最底层的 $1$ 到 $n$ 个节点的子树,分别代表第 $1$ 到第 $n$ 行的线段树。那么这些底层的节点对应的父节点,就代表其两个子节点的子树所在的一片区域。

\subsubsection{空间复杂度}

通常情况下,我们不可能对于外层线段树的每一个结点都建立一颗子线段树,空间需求过大。树套树一般采取动态开点的策略。单次修改,我们会涉及到外层线段树的 $\log{n}$ 个节点,且对于每个节点的子树涉及 $\log{n}$ 个节点,所以单次修改产生的空间最多为 $\log^2{n}$。

\subsubsection{时间复杂度}

对于询问操作,我们考虑我们在外层线段树上进行 $\log{n}$ 次操作,每次操作会在一个内层线段树上进行 $\log{n}$ 次操作,所以时间复杂度为 $\log^2{n}$。

修改操作,与询问操作复杂度相同,也为 $\log^2{n}$。

\subsubsection{经典例题}

\href{https://www.lydsy.com/JudgeOnline/problem.php?id=3262}{陌上花开} 将第一维排序处理,然后用树套树维护第二维和第三维。

\subsubsection{示例代码}

第二维查询

\begin{cppcode}
int tree_query(int k, int l, int r, int x) {
  if (k == 0) return 0;
  if (1 <= l && r <= sec[x].y) return vec_query(ou_root[k], 1, p, 1, sec[x].z);
  int mid = l + r >> 1, res = 0;
  if (1 <= mid) res += tree_query(ou_ch[k][0], l, mid, x);
  if (sec[x].y > mid) res += tree_query(ou_ch[k][1], mid + 1, r, x);
  return res;
}
\end{cppcode}

第二维修改

\begin{cppcode}
void tree_insert(int &k, int l, int r, int x) {
  if (k == 0) k = ++ou_tot;
  vec_insert(ou_root[k], 1, p, sec[x].z);
  if (l == r) return;
  int mid = l + r >> 1;
  if (sec[x].y <= mid)
    tree_insert(ou_ch[k][0], l, mid, x);
  else
    tree_insert(ou_ch[k][1], mid + 1, r, x);
}
\end{cppcode}

第三维查询

\begin{cppcode}
int vec_query(int k, int l, int r, int x, int y) {
  if (k == 0) return 0;
  if (x <= l && r <= y) return data[k];
  int mid = l + r >> 1, res = 0;
  if (x <= mid) res += vec_query(ch[k][0], l, mid, x, y);
  if (y > mid) res += vec_query(ch[k][1], mid + 1, r, x, y);
  return res;
}
\end{cppcode}

第三维修改

\begin{cppcode}
void vec_insert(int &k, int l, int r, int loc) {
  if (k == 0) k = ++tot;
  data[k]++;
  if (l == r) return;
  int mid = l + r >> 1;
  if (loc <= mid) vec_insert(ch[k][0], l, mid, loc);
  if (loc > mid) vec_insert(ch[k][1], mid + 1, r, loc);
}
\end{cppcode}

\subsubsection{相关算法}

面对多维度信息的题目时,如果题目没有要求强制在线,我们还可以考虑 \textbf{CDQ 分治},或者\textbf{整体二分}等分治算法,来避免使用高级数据结构,减少代码实现难度。
