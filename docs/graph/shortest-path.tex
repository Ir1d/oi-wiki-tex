
\subsection{定义}

(还记得这些定义吗?在阅读下列内容之前,请务必了解  图论基础  部分。)

\begin{itemize}
\item 路径
\item 最短路
\item 有向图中的最短路、无向图中的最短路
\item 单源最短路、每对结点之间的最短路
\end{itemize}

\subsection{性质}

对于边权为正的图,任意两个结点之间的最短路,不会经过重复的结点。

对于边权为正的图,任意两个结点之间的最短路,不会经过重复的边。

对于边权为正的图,任意两个结点之间的最短路,任意一条的结点数不会超过 $n$,边数不会超过 $n-1$。

\subsection{Floyd 算法}

是用来求任意两个结点之间的最短路的。

复杂度比较高,但是常数小,容易实现。(我会说只有三个 \texttt{for} 吗?)

适用于任何图,不管有向无向,边权正负,但是最短路必须存在。(不能有个负环)

\subsubsection{实现}

我们定义一个数组 \texttt{f[k][x][y]},表示只允许经过结点 $1$ 到 $k$,结点 $x$ 到结点 $y$ 的最短路长度。

很显然,\texttt{f[n][x][y]} 就是结点 $x$ 到结点 $y$ 的最短路长度。

我们来考虑怎么求这个数组

\texttt{f[0][x][y]}:边权,或者 $0$,或者 $+\infty$ (\texttt{f[0][x][x]} 什么时候应该是 $+\infty$?)

\texttt{f[k][x][y] = min(f[k-1][x][y], f[k-1][x][k]+f[k-1][k][y])}

上面两行都显然是对的,然而这个做法空间是 $O(N^3)$。

但我们发现数组的第一维是没有用的,于是可以直接改成 \texttt{f[x][y] = min(f[x][y], f[x][k]+f[k][y])},

\begin{cppcode}
for (k = 1; k <= n; k++) {
  for (i = 1; i <= n; i++) {
    for (j = 1; j <= n; j++) {
      f[i][j] = min(f[i][j], f[i][k] + f[k][j]);
    }
  }
}
\end{cppcode}

时间复杂度是 $O(N^3)$,空间复杂度是 $O(N^2)$。

\subsubsection{应用}

\begin{QUESTION}{给一个正权无向图,找一个最小权值和的环。}{}
首先这一定是一个简单环。

想一想这个环是怎么构成的。

考虑环上编号最大的结点 u。

\texttt{f[u-1][x][y]} 和 (u,x), (u,y)共同构成了环。

在Floyd的过程中枚举u,计算这个和的最小值即可。

$O(n^3)$。

\end{QUESTION}


\begin{QUESTION}{已知一个有向图中任意两点之间是否有连边,要求判断任意两点是否联通。}{}

该问题即是求\textbf{图的传递闭包}。

我们只需要按照 Floyd 的过程,逐个加入点判断一下。

只是此时的边的边权变为 $1/0$, 而取 $\min$ 变成了\textbf{与}运算。

再进一步用 bitset 优化,复杂度可以到 $O(\frac{n^3}{w})$。

\begin{cppcode}
//std::bitset<SIZE> f[SIZE];
for (k = 1; k <= n; k++)
  for (i = 1; i <= n; i++)
      if(f[i][k]) f[i] = f[i] & f[k];
\end{cppcode}

\end{QUESTION}


\hr

\subsection{Bellman-Ford 算法}

一种基于松弛(relax)操作的最短路算法。

支持负权。

能找到某个结点出发到所有结点的最短路,或者报告某些最短路不存在。

在国内 OI 界,你可能听说过的 “SPFA”,就是 Bellman-Ford 算法的一种实现。(优化)

\subsubsection{实现}

假设结点为 $S$。

先定义 $dist(u)$ 为 $S$ 到 $u$ (当前)的最短路径长度。

$relax(u,v)$: $dist(v) = min(dist(v), dist(u) + edge\_len(u, v))$.

$relax$ 是从哪里来的呢?

三角形不等式: $dist(v) \leq dist(u) + edge\_len(u, v)$。

证明:反证法,如果不满足,那么可以用 $relax$ 操作来更新 $dist(v)$ 的值。

Bellman-Ford 算法如下:

\begin{minted}{text}
while (1) for each edge(u, v) relax(u, v);
\end{minted}

当一次循环中没有 $relax$ 操作成功时停止。

每次循环是 $O(m)$ 的,那么最多会循环多少次呢?

答案是 $\infty$!(如果有一个 $S$ 能走到的负环就会这样)

但是此时某些结点的最短路不存在。

我们考虑最短路存在的时候。

由于一次 $relax$ 会使(被 $relax$ 的)最短路的边数至少 $+1$,而最短路的边数最多为 $n-1$。

所以最多(连续)$relax$ $n-1$ 次……($relax$ 一定是环环相扣的,不然之前就能被 $relax$ 掉)

所以最多循环 $n-1$ 次。

总时间复杂度 $O(NM)$。 \textbf{(对于最短路存在的图)}

\begin{minted}{text}
relax(u, v) {
  dist[v] = min(dist[v], dist[u] + edge_len(u, v));
}
for (i = 1; i <= n; i++) {
  dist[i] = edge_len(S, i);
}
for (i = 1; i < n; i++) {
  for each edge(u, v) {
    relax(u, v);
  }
}
\end{minted}

注:这里的 $edge\_len(u, v)$ 表示边的权值,如果该边不存在则为 $+\infty$,$u=v$ 则为 $0$。

\subsubsection{应用}

给一张有向图,问是否存在负权环。

做法很简单,跑 Bellman-Ford 算法,如果有个点被 $relax$ 成功了 $n$ 次,那么就一定存在。

如果 $n-1$ 次之内算法结束了,就一定不存在。

\subsubsection{队列优化:SPFA}

即 Shortest Path Faster Algorithm。

很多时候我们并不需要那么多无用的 $relax$ 操作。

很显然,只有上一次被 $relax$ 的结点,所连接的边,才有可能引起下一次的 $relax$。

那么我们用队列来维护 “哪些结点可能会引起 $relax$”,就能只访问必要的边了。

\begin{minted}{text}
q = new queue();
q.push(S);
in_queue[S] = true;
while (!q.empty()) {
  u = q.pop();
  in_queue[u] = false;
  for each edge(u, v) {
    if (relax(u, v) && !in_queue[v]) {
      q.push(v);
      in_queue[v] = true;
    }
  }
}
\end{minted}

SPFA 的时间复杂度为 $O(kM)~ (k\approx 2)$ (玄学),但 \textbf{理论上界} 为 $O(NM)$,精心设计的稠密图可以随便卡掉 SPFA,所以考试时谨慎使用  (NOI 2018 卡 SPFA)。

\paragraph{SPFA 的优化之 SLF}

即 Small Label First。

即在新元素加入队列时,如果队首元素权值大于新元素权值,那么就把新元素加入队首,否则依然加入队尾。

该优化在确实在一些图上有显著效果,其复杂度也有保证,但是如果有负权边的话,可以直接卡到指数级。

\hr

\subsection{Dijkstra 算法}

Dijkstra 是个人名(荷兰姓氏)。

IPA: /ˈdikstrɑ/ 或 /ˈdɛikstrɑ/。

这种算法只适用于非负权图,但是时间复杂度非常优秀。

也是用来求单源最短路径的算法。

\subsubsection{实现}

主要思想是,将结点分成两个集合:已确定最短路长度的,未确定的。

一开始第一个集合里只有 $S$。

然后重复这些操作:

(1)$relax$ 那些刚刚被加入第一个集合的结点的所有出边。

(2)从第二个集合中,选取一个最短路长度最小的结点,移到第一个集合中。

直到第二个集合为空,算法结束。

时间复杂度:只用分析集合操作,$n$ 次 \texttt{delete-min},$m$ 次 \texttt{decrease-key}。

如果用暴力: $O(n^2 + m)$。

如果用堆:$O((n+m) \log m)$。

如果用线段树(ZKW 线段树):$(O(n+m)\log n)$

如果用 Fibonacci 堆: $O(n \log n + m)$(这就是为啥优秀了)。

等等,还没说正确性呢!

分两步证明:先证明任何时候第一个集合中的元素的 $dist$ 一定不大于第二个集合中的。

再证明第一个集合中的元素的最短路已经确定。

第一步,一开始时成立(基础),在每一步中,加入集合的元素一定是最大值,且是另一边最小值,$relax$ 又是加上非负数,所以仍然成立。(归纳) (利用非负权值的性质)

第二步,考虑每次加进来的结点,到他的最短路,上一步必然是第一个集合中的元素(否则他不会是第二个集合中的最小值,而且有第一步的性质),又因为第一个集合已经全部 $relax$ 过了,所以最短路显然确定了。

\begin{minted}{text}
H = new heap();
H.insert(S, 0);
dist[S] = 0;
for (i = 1; i <= n; i++) {
  u = H.delete_min();
  for each edge(u, v) {
    if (relax(u, v)) {
      H.decrease_key(v, dist[v]);
    }
  }
}
\end{minted}

\hr

\subsection{不同方法的比较}

\begin{tabular}{ccc}
\hline
Floyd& Bellman-Ford& Dijkstra\\每对结点之间的最短路& 单源最短路& 单源最短路\\没有负环的图& 任意图& 非负权图\\O(N\textasciicircum{}3)& O(NM)& O((N+M)\textbackslash{}log M)\\\hline
\end{tabular}

\subsection{拓展:分层图最短路}

分层图最短路,一般模型为有 $k$ 次零代价通过一条路径,求总的最小花费。对于这种题目,我们可以采用 DP 相关的思想,设 $\text{dis}_{i, j}$ 表示当前从起点 $i$ 号结点,使用了 $j$ 次免费通行权限后的最短路径。显然,$\text{dis}$ 数组可以这么转移:

$\text{dis}_{i, j} = \min\{\min\{\text{dis}_{from, j - 1}\}, \min\{\text{dis}_{from,j} + w\}\}$

其中,$from$ 表示 $i$ 的父亲节点,$w$ 表示当前所走的边的边权。当 $j - 1 \geq k$ 时,$\text{dis}_{from, j}$ = $\infty$。

事实上,这个 DP 就相当于把每个结点拆分成了 $k+1$ 个结点,每个新结点代表使用不同多次免费通行后到达的原图结点。换句话说,就是每个结点 $u_i$ 表示使用 $i$ 次免费通行权限后到达 $u$ 结点。

\subsubsection{模板题:\href{https://www.luogu.org/problemnew/show/P4568}{JLOI2011 飞行路线}}

题意:有一个 $n$ 个点 $m$ 条边的无向图,你可以选择 $k$ 条道路以零代价通行,求 $s$ 到 $t$ 的最小花费。

参考核心代码:

\begin{cppcode}
struct State {    // 优先队列的结点结构体
  int v, w, cnt;  // cnt 表示已经使用多少次免费通行权限
  State() {}
  State(int v, int w, int cnt) : v(v), w(w), cnt(cnt) {}
  bool operator<(const State &rhs) const { return w > rhs.w; }
};

void dijkstra() {
  memset(dis, 0x3f, sizeof dis);
  dis[s][0] = 0;
  pq.push(State(s, 0, 0));  // 到起点不需要使用免费通行权,距离为零
  while (!pq.empty()) {
    const State top = pq.top();
    pq.pop();
    int u = top.v, nowCnt = top.cnt;
    if (done[u][nowCnt]) continue;
    done[u][nowCnt] = true;
    for (int i = head[u]; i; i = edge[i].next) {
      int v = edge[i].v, w = edge[i].w;
      if (nowCnt < k && dis[v][nowCnt + 1] > dis[u][nowCnt]) {  // 可以免费通行
        dis[v][nowCnt + 1] = dis[u][nowCnt];
        pq.push(State(v, dis[v][nowCnt + 1], nowCnt + 1));
      }
      if (dis[v][nowCnt] > dis[u][nowCnt] + w) {  // 不可以免费通行
        dis[v][nowCnt] = dis[u][nowCnt] + w;
        pq.push(State(v, dis[v][nowCnt], nowCnt));
      }
    }
  }
}

int main() {
  n = read(), m = read(), k = read();
  // 笔者习惯从 1 到 n 编号,而这道题是从 0 到 n - 1,所以要处理一下
  s = read() + 1, t = read() + 1;
  while (m--) {
    int u = read() + 1, v = read() + 1, w = read();
    add(u, v, w), add(v, u, w);  // 这道题是双向边
  }
  dijkstra();
  int ans = std::numeric_limits<int>::max();  // ans 取 int 最大值为初值
  for (int i = 0; i <= k; ++i)
    ans = std::min(ans, dis[t][i]);  // 对到达终点的所有情况取最优值
  println(ans);
}
\end{cppcode}
