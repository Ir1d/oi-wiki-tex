
Kirchhoff 矩阵树定理(简称矩阵树定理)解决了一张图的生成树个数计数问题。

\subsection{本篇记号声明}

\textbf{本篇中的图,无论无向还是有向,都允许重边,但是不允许自环。}

\subsubsection{无向图情况}

设 $G$ 是一个有 $n$ 个顶点的无向图。定义度数矩阵 $D(G)$ 为:

$$
_{ii}(G) = \mathrm{deg}(i), D_{ij} = 0, i\neq j
$$

设 $\#e(i,j)$ 为点 $i$ 与点 $j$ 相连的边数,并定义邻接矩阵 $A$ 为:

$$
A_{ij}(G)=\#e(i,j), i\neq j
$$

定义 Laplace 矩阵(亦称 Kirchhoff 矩阵)$L$为:

$$
L(G) = D(G) - A(G)
$$

记图 $G$ 的所有生成树个数为 $t(G)$。

\subsubsection{有向图情况}

设 $G$ 是一个有 $n$ 个顶点的有向图。定义出度矩阵 $D^{out}(G)$ 为:

$$
D^{out}_{ii}(G) = \mathrm{deg^{out}}(i), D^{out}_{ij} = 0, i\neq j
$$

类似地定理入度矩阵 $D^{in}(G)$

设 $\#e(i,j)$ 为点 $i$ 指向点 $j$ 的有向边数,并定义邻接矩阵 $A$ 为:

$$
A_{ij}(G)=\#e(i,j), i\neq j
$$

定义出度 Laplace 矩阵 $L^{out}$ 为:

$$
L^{out}(G) = D^{out}(G) - A(G)
$$

类似地定义入度 Laplace 矩阵 $L^{in}$。

记图 $G$ 的以 $r$ 为根的所有根向树形图个数为 $t^{root}(G,r)$。所谓根向树形图,是说这张图的基图是一棵树,所有的边全部指向父亲。

记图 $G$ 的以 $r$ 为根的所有叶向树形图个数为 $t^{leaf}(G,r)$。所谓叶向树形图,是说这张图的基图是一棵树,所有的边全部指向儿子。

\subsection{定理叙述}

矩阵树定理具有多种形式。其中用得较多的是定理 1、定理 3 与定理 4。

\textbf{定理 1 (矩阵树定理,无向图行列式形式)} 对于任意的 $i$,都有

$$
t(G) = \det L(G)\binom{1,2,\cdots,i-1,i+1,\cdots,n}{1,2,\cdots,i-1,i+1,\cdots,n}
$$

其中记号 $L(G)\binom{1,2,\cdots,i-1,i+1,\cdots,n}{1,2,\cdots,i-1,i+1,\cdots,n}$ 表示矩阵 $L(G)$ 的第 $1,\cdots,i-1,i+1,\cdots,n$ 行与第 $1,\cdots,i-1,i+1,\cdots,n$ 列构成的子矩阵。也就是说,无向图的 Laplace 矩阵具有这样的性质,它的所有 $n-1$ 阶主子式都相等。

\textbf{定理 2 (矩阵树定理,无向图特征值形式)} 设 $\lambda_1, \lambda_2, \cdots, \lambda_{n-1}$ 为 $L(G)$ 的 $n - 1$ 个非零特征值,那么有

$t(G) = \frac{1}{n}\lambda_1\lambda_2\cdots\lambda_{n-1}$

\textbf{定理 3 (矩阵树定理,有向图根向形式)} 对于任意的 $k$,都有

$$
t^{root}(G,k) = \det L^{out}(G)\binom{1,2,\cdots,k-1,k+1,\cdots,n}{1,2,\cdots,k-1,k+1,\cdots,n}
$$

因此如果要统计一张图所有的根向树形图,只要枚举所有的根 $k$ 并对 $t^{root}(G,k)$ 求和即可。

\textbf{定理 4 (矩阵树定理,有向图叶向形式)} 对于任意的 $k$,都有

$$
t^{leaf}(G,k) = \det L^{in}(G)\binom{1,2,\cdots,k-1,k+1,\cdots,n}{1,2,\cdots,k-1,k+1,\cdots,n}
$$

因此如果要统计一张图所有的叶向树形图,只要枚举所有的根 $k$ 并对 $t^{leaf}(G,k)$ 求和即可。

\subsection{BEST 定理}

\textbf{定理 5 (BEST 定理)} 设$G$是有向欧拉图,那么$G$的不同欧拉回路总数$ec(G)$是

$$
ec(G) = t^{root}(G,k)\prod_{v\in V}(\deg (v) - 1)!
$$

注意,对欧拉图$G$的任意两个节点$k, k'$,都有$t^{root}(G,k)=t^{root}(G,k')$,且欧拉图$G$的所有节点的入度和出度相等。

\subsection{例题}

\begin{NOTE}{例题 1}{}
HEOI2015: 小 Z 的房间,请参考\url{https://www.lydsy.com/JudgeOnline/problem.php?id=4031}

\end{NOTE}


\textbf{解} 矩阵树定理的裸题。将每个空房间看作一个结点,根据输入的信息建图,得到 Laplace 矩阵后,任意删掉 L 的第 $i$ 行第 $i$ 列,求这个子式的行列式即可。求行列式的方法就是高斯消元成上三角阵然后算对角线积。另外本题需要在模 $k$ 的整数子环 $\mathbb{Z}_k$ 上进行高斯消元,采用辗转相除法即可。

\begin{NOTE}{例题 2}{}
FJOI2007: 轮状病毒。请参考\url{https://www.lydsy.com/JudgeOnline/problem.php?id=1002}

\end{NOTE}


\textbf{解} 本题的解法很多,这里用矩阵树定理是最直接的解法。当输入为 $n$ 时,容易写出其 $n+1$ 阶的 Laplace 矩阵为:

$$
L_n = \begin{bmatrix}
n&  -1&  -1&  -1&  \cdots&  -1&  -1\\
-1&  3&  -1&  0&  \cdots&  0&  -1\\
-1&  -1&  3&  -1&  \cdots&  0&  0\\
-1&  0&  -1&  3&  \cdots&  0&  0\\
\vdots&  \vdots&  \vdots&  \vdots&  \ddots&  \vdots&  \vdots\\
-1&  0&  0&  0&  \cdots&  3&  -1\\
-1&  -1&  0&  0&  \cdots&  -1&  3\\
\end{bmatrix}_{n+1}
$$

求出它的 $n$ 阶子式的行列式即可,剩下的只有高精度计算了。

\begin{NOTE}{例题 2+}{}
将例题 2 的数据加强,要求 $n\leq 100000$,但是答案对 1000007 取模。(本题求解需要一些线性代数知识)

\end{NOTE}


\textbf{解} 推导递推式后利用矩阵快速幂即可求得。

\begin{DANGER}{推导递推式的过程。警告:过程冗杂}{}

注意到 $L_n$ 删掉第 1 行第 1 列以后得到的矩阵很有规律,因此其实就是在求矩阵

$$
M_n = \begin{bmatrix}
3&  -1&  0&  \cdots&  0&  -1\\
-1&  3&  -1&  \cdots&  0&  0\\
0&  -1&  3&  \cdots&  0&  0\\
\vdots&  \vdots&  \vdots&  \ddots&  \vdots&  \vdots\\
0&  0&  0&  \cdots&  3&  -1\\
-1&  0&  0&  \cdots&  -1&  3\\
\end{bmatrix}_{n}
$$

的行列式。对 $M_n$ 的行列式按第一列展开,得到

$$
\det M_n = 3\det \begin{bmatrix}
3&  -1&  \cdots&  0&  0\\
-1&  3&  \cdots&  0&  0\\
\vdots&  \vdots&  \ddots&  \vdots&  \vdots\\
0&  0&  \cdots&  3&  -1\\
0&  0&  \cdots&  -1&  3\\
\end{bmatrix}_{n-1} + \det\begin{bmatrix}
-1&  0&  \cdots&  0&  -1\\
-1&  3&  \cdots&  0&  0\\
\vdots&  \vdots&  \ddots&  \vdots&  \vdots\\
0&  0&  \cdots&  3&  -1\\
0&  0&  \cdots&  -1&  3\\
\end{bmatrix}_{n-1} + (-1)^n \det\begin{bmatrix}
-1&  0&  \cdots&  0&  -1\\
3&  -1&  \cdots&  0&  0\\
-1&  3&  \cdots&  0&  0\\
\vdots&  \vdots&  \ddots&  \vdots&  \vdots\\
0&  0&  \cdots&  3&  -1\\
\end{bmatrix}_{n-1}
$$

上述三个矩阵的行列式记为 $d_{n-1}, a_{n-1}, b_{n-1}$。注意到 $d_n$ 是三对角行列式,采用类似的展开的方法可以得到 $d_n$ 具有递推公式 $d_n=3d_{n-1}-d_{n-2}$。类似地,采用展开的方法可以得到 $a_{n-1}=-d_{n-2}-1$,以及 $(-1)^n b_{n-1}=-d_{n-2}-1$。将这些递推公式代入上式,得到

$\det M_n = 3d_{n-1}-2d_{n-2}-2$

$d_n = 3d_{n-1}-d_{n-2}$

于是猜测 $\det M_n$ 也是非齐次的二阶线性递推。采用待定系数法可以得到最终的递推公式为

$\det M_n = 3\det M_{n-1} - \det M_{n-2} + 2$

改写成 $(\det M_n+2) = 3(\det M_{n-1}+2) - (\det M_{n-2} + 2)$ 后,采用矩阵快速幂即可求出答案。

\end{DANGER}


\begin{NOTE}{例题 3}{}
BZOJ3659: WHICH DREAMED IT

\end{NOTE}


\textbf{解} 本题是 BEST 定理的直接应用,但是要注意,由于题目规定 “两种完成任务的方式算作不同当且仅当使用钥匙的顺序不同”,对每个欧拉回路,1 号房间可以沿着任意一条出边出发,从而答案还要乘以 1 号房间的出度。

\subsection{注释}

根向树形图在一些地方被称为内向树形图,但因为计算内向树形图用的是出度,为了不引起 in 和 out 的混淆,所以采用了根向这一说法。
