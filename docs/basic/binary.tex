
\subsection{二分搜索}

二分搜索,也称折半搜索,是用来在一个有序数组中查找某一元素的算法。

以在一个升序数组中查找一个数为例。

它每次考察数组当前部分的中间元素,如果中间元素刚好是要找的,就结束搜索过程;如果中间元素小于所查找的值,那么左侧的只会更小,不会有所查找的元素,只需要到右侧去找就好了;如果中间元素大于所查找的值,同理,右侧的只会更大而不会有所查找的元素,所以只需要到左侧去找。

在二分搜索过程中,每次都把查询的区间减半,因此对于一个长度为 $n$ 的数组,至多会进行 $O(\log n)$ 次查找。

\begin{cppcode}
int binary_search(int start, int end, int key) {
  int ret = -1;  // 未搜索到数据返回-1下标
  int mid;
  while (start <= end) {
    mid = start + ((end - start) >> 1);  //直接平均可能会溢出,所以用这个算法
    if (arr[mid] < key)
      start = mid + 1;
    else if (arr[mid] > key)
      end = mid - 1;
    else {  // 最后检测相等是因为多数搜索情况不是大于就是小于
      ret = mid;
      break;
    }
  }
  return ret;  // 单一出口
}
\end{cppcode}

\begin{NOTE}{}{}
\texttt{>> 1} 比 \texttt{/ 2} 速度快一些

\end{NOTE}


注意,这里的有序是广义的有序,如果一个数组中的左侧或者右侧都满足某一种条件,而另一侧都不满足这种条件,也可以看作是一种有序(如果把满足条件看做 $1$,不满足看做 $0$,至少对于这个条件的这一维度是有序的)。换言之,二分搜索法可以用来查找满足某种条件的最大(最小)的值。

如果我们要求满足某种条件的最大值的最小可能情况(最大值最小化)呢?首先的想法是从小到大枚举这个作为答案的「最大值」,然后去判断是否合法。要是这个答案是单调的就好了,那样就可以使用二分搜索法来更快地找到答案。

要想使用二分搜索法来解这种「最大值最小化」的题目,需要满足以下三个条件:

\begin{enumerate}
\item 答案在一个固定区间内;
\item 可能查找一个符合条件的值不是很容易,但是要求能比较容易地判断某个值是否是符合条件的;
\item 可行解对于区间满足一定的单调性。换言之,如果 $x$ 是符合条件的,那么有 $x + 1$ 或者 $x - 1$ 也符合条件。(这样下来就满足了上面提到的单调性)
\end{enumerate}

当然,最小值最大化是同理的。

二分法把一个寻找极值的问题转化成一个判定的问题(用二分搜索来找这个极值)。类比枚举法,我们当时是枚举答案的可能情况,现在由于单调性,我们不再需要一个个枚举,利用二分的思路,就可以用更优的方法解决「最大值最小」、「最小值最大」。这种解法也成为是「二分答案」,常见于解题报告中。

\subsubsection{STL 的二分查找}

补充一个小知识点, 对于一个有序的 array 你可以使用 \texttt{std::lower\_bound()} 来找到第一个大于等于你的值的数, \texttt{std::upper\_bound()} 来找到第一个大于你的值的数。

请注意,必须是有序数组,否则答案是错误的。

关于具体使用方法,请参见  STL 页面 。

\subsection{三分法}

\begin{cppcode}
mid = left + (right - left >> 1);
midmid = mid + (right - mid >> 1);  // 对右侧区间取半
if (cal(mid) > cal(midmid))
  right = midmid;
else
  left = mid;
\end{cppcode}

三分法可以用来查找凸函数的最大(小)值。

画一下图好理解一些(图待补)

\begin{itemize}
\item 如果 \texttt{mid} 和 \texttt{midmid} 在最大(小)值的同一侧:
那么由于单调性,一定是二者中较大(小)的那个离最值近一些,较远的那个点对应的区间不可能包含最值,所以可以舍弃。
\item 如果在两侧:
由于最值在二者中间,我们舍弃两侧的一个区间后,也不会影响最值,所以可以舍弃。
\end{itemize}

\subsection{分数规划}

分数规划是这样一类问题,每个物品有两个代价 $c_i$,$d_i$,要求通过某种方式选出若干个,使得 $\frac{\sum{c_i}}{\sum{d_i}}$ 最大或最小。

经典的例子是 最优比率环、最优比率生成树 等等。

\subsubsection{二分法}

比如说我们要求的是最小的,记 $L$ 为最优的答案,对这个式子做一些变换:

$$
L \geq \frac{\sum{c_i}}{\sum{d_i}}
$$

把分母乘过去,把右侧化为 $0$:

$$
{\sum{d_i}} \times L - {\sum{c_i}} \geq 0
$$

即:

$$
{\sum_{i=1}^N{d_i}} \times L - {\sum_{i=1}^N{c_i}} \geq 0
$$

$$
\sum_{i=1}^N{d_i \times L - c_i} \geq 0
$$

不难发现,如果 $L'$ 比 $L$ 要小,上式左端的值会更大一些。

所以要求得最小的 $L$,我们要求的就变成了让上式左端最接近 $0$ 的 $L$。

不难发现左端的式子是随 $L$ 变化而单调变化的,所以可以通过二分法来解决。

\subsubsection{Dinkelbach 算法}

Dinkelbach 算法是每次用上一轮的答案当做新的 $L$ 来输入,不断地迭代,直至答案收敛。
