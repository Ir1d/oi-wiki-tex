
模拟。顾名思义,就是用计算机来模拟题目中要求的操作,比如 NOIP 2014 的 \href{https://loj.ac/problem/2498}{生活大爆炸版石头剪刀布},只需要按照题面的意思来写就可以了。

当然,模拟一般也不是很好写,参见经典题目 \href{http://bailian.openjudge.cn/practice/3750/}{魔兽世界} 和 \href{https://www.lydsy.com/JudgeOnline/problem.php?id=1972}{猪国杀}。

模拟题目通常具有码量大、操作多、思路繁复的特点。并且由于它码量大,会导致很难查错,如果是在考试上是相当浪费时间的。

所以写模拟题,遵循以下的建议有可能会帮助你减少浪费时间

\begin{enumerate}
\item 在动手写代码之前,在草纸上尽可能的写好要实现的流程
\item 在代码中,尽量把每个部分模块化、写成函数、结构体或类
\item 对于一些可能重复用到的概念,可以统一转化,方便处理 : 如,某题给你 "YY-MM-DD 时: 分" 把它扔到一个函数处理成秒,会减少概念混淆
\item 调试时分块调试,模块化的好处就是可以方便的单独调某一部分
\item 写代码的时候一定要思路清晰,不要想到什么写什么,要按照落在纸上的步骤写。
\end{enumerate}
