
\subsection{文件的概念}

文件是根据特定的目的而收集在一起的有关数据的集合。C/C++ 把每一个文件都看成是一个有序的字节流,每个文件都是以\textbf{文件结束标志}(EOF)结束,如果要操作某个文件,程序应该首先打开该文件,每当一个文件被打开后(请记得关闭打开的文件),该文件就和一个流关联起来,这里的流实际上是一个字节序列。  

C/C++ 将文件分为文本文件和二进制文件。文本文件就是简单的文本文件(重点),另外二进制文件就是特殊格式的文件或者可执行代码文件等。

\subsection{文件的操作步骤}

 1、打开文件,将文件指针指向文件,决定打开文件类型;  

 2、对文件进行读、写操作(比赛中主要用到的操作,其他一些操作暂时不写);  

 3、在使用完文件后,关闭文件。  

\subsection{\texttt{freopen} 函数}

\subsubsection{命令格式}

\begin{cppcode}
FILE* freopen(const char* filename, const char* mode, FILE* stream);
\end{cppcode}

\subsubsection{参数说明}

\begin{itemize}
\item \texttt{filename}: 要打开的文件名
\item \texttt{mode}: 文件打开的模式
\item \texttt{stream}: 文件指针,通常使用标准文件流 (\texttt{stdin/stdout/stderr})  
\end{itemize}

\subsubsection{使用方法}

读入文件内容:

\begin{cppcode}
freopen("data.in", "r", stdin);
// data.in 就是读取的文件名,要和可执行文件放在同一目录下
\end{cppcode}

输出到文件:

\begin{cppcode}
freopen("data.out", "w", stdout);
// data.out 就是输出文件的文件名,和可执行文件在同一目录下
\end{cppcode}

关闭标准输入 / 输出流  

\begin{cppcode}
fclose(stdin);
fclose(stdout);
\end{cppcode}

\subsubsection{模板}

\begin{cppcode}
#include <cstdio>
#include <iostream>
int main(void) {
  freopen("data.in", "r", stdin);
  freopen("data.out", "w", stdout);
  /*
  中间的代码不需要改变,直接使用 cin 和 cout 即可
  */
  fclose(stdin);
  fclose(stdout);
  return 0;
}
\end{cppcode}

参考书目:信息学奥赛一本通

\subsection{C++ 的 ifstream/ofstream 文件输入输出流}

\subsubsection{使用方法}

读入文件内容:

\begin{cppcode}
ifstream fin("data.in");
// data.in 就是读取的文件名,要和可执行文件放在同一目录下
\end{cppcode}

输出到文件:

\begin{cppcode}
ofstream fout("data.out");
// data.out 就是输出文件的文件名,和可执行文件在同一目录下
\end{cppcode}

关闭标准输入  输出流

\begin{cppcode}
fin.close();
fout.close();
\end{cppcode}

\subsubsection{模板}

\begin{cppcode}
#include <cstdio>
#include <fstream>
ifstream fin("data.in");
ofstream fout("data.out");
int main(void) {
  /*
  中间的代码改变 cin 为 fin ,cout 为 fout 即可
  */
  fin.close();
  fout.close();
  return 0;
}
\end{cppcode}
