
\subsection{费马小定理}

若 $p$ 为素数,$\gcd(a, p) = 1$,则 $a^{p - 1} \equiv 1 \pmod{p}$。

另一个形式:对于任意整数 $a$,有 $a^p \equiv a \pmod{p}$。

\subsection{欧拉定理}

若 $\gcd(a, m) = 1$,则 $a^{\phi(m)} \equiv 1 \pmod{m}$。

\subsubsection{证明}

设 $r_1, r_2, \cdots, r_{\phi(m)}$ 为模 $m$ 意义下的一个简化剩余系,则 $ar_1, ar_2, \cdots, ar_{\phi(m)}$ 也为模 $m$ 意义下的一个简化剩余系。所以 $r_1r_2 \cdots r_{\phi(m)} \equiv ar_1 \cdot ar_2 \cdots ar_{\phi(m)} \equiv a^{\phi(m)}r_1r_2 \cdots r_{\phi(m)} \pmod{m}$,可约去 $r_1r_2 \cdots r_{\phi(m)}$,即得 $a^{\phi(m)} \equiv 1 \pmod{m}$。

当 $m$ 为素数时,由于 $\phi(m) = m - 1$,代入欧拉定理可立即得到费马小定理。

\subsection{扩展欧拉定理}

$$
a^b\equiv
\begin{cases}
a^{b\bmod\varphi(p)},\,&\gcd(a,\,p)=1\\
a^b,&\gcd(a,\,p)\ne1,\,b<\varphi(p)\\
a^{b\bmod\varphi(p)+\varphi(p)},&\gcd(a,\,p)\ne1,\,b\ge\varphi(p)
\end{cases}
\pmod p
$$

\subsubsection{证明}

证明转载自 \href{http://blog.csdn.net/synapse7/article/details/19610361}{synapse7}

\begin{enumerate}
\item 在 $a$ 的 $0$ 次,$1$ 次,...,$b$ 次幂模 $m$ 的序列中,前 $r$ 个数($a^0$ 到 $a^{r-1}$) 互不相同,从第 $r$ 个数开始,每 $s$ 个数就循环一次。
证明:由鸽巢定理易证。
我们把 $r$ 称为 $a$ 幂次模 $m$ 的循环起始点,$s$ 称为循环长度。(注意:$r$ 可以为 $0$)
用公式表述为:$a^r\equiv a^{r+s}\pmod{m}$ 
\item $a$ 为素数的情况
令 $m=p^rm'$,则 $\gcd(p,m')=1$,所以 $p^{\phi(m')}\equiv 1\pmod{m'}$ 
又由于 $\gcd(p^r,m')=1$,所以 $\phi(m') \mid \varphi(m)$,所以 $p^{\varphi(m)}\equiv 1 \pmod {m'}$,即 $p^\phi(m)=km'+1$,两边同时乘以 $p^r$,得 $p^{r+\phi(m)}=km+p^r$(因为 $m=p^rm'$ )
所以 $p^r\equiv p^{r+s}\pmod m$,这里 $s=\phi(m)$
\item 推论:$p^b\equiv p^{r+(b-r) \mod \phi(m)}\pmod m$ 
\item 又由于 $m=p^rm'$,所以 $\phi(m) \ge  \phi(p^r)=p^{r-1}(p-1) \ge r$ 
所以 $p^r\equiv p^{r+\phi(m)}\equiv p^{r \mod \phi(m)+\phi(m)}\pmod m$ 
所以 $p^b\equiv p^{r+(b-r) \mod \phi(m)}\equiv p^{r \mod \phi(m)+\phi(m)+(b-r) \mod \phi(m)}\equiv p^{\phi(m)+b \mod \phi(m)}\pmod m$ 
即 $p^b\equiv p^{b \mod \phi(m)+\phi(m)}\pmod m$ 
\item $a$ 为素数的幂的情况
是否依然有 $a^{r'}\equiv a^{r'+s'}\pmod m$?(其中 $s'=\phi(m),a=p^k$)
答案是肯定的,由 2 知 $p^s\equiv 1 \pmod m'$,所以 $p^{s \times \frac{k}{\gcd(s,k)}} \equiv 1\pmod {m'}$,所以当 $s'=\frac{s}{\gcd(s,k)}$ 时才能有 $p^{s'k}\equiv 1\pmod {m'}$ ,此时 $s' \mid s \mid \phi(m)$ ,且 $r'= \lceil \frac{r}{k}\rceil \le r \le \phi(m)$ ,由 $r',s'$ 与 $\phi(m)$ 的关系,依然可以得到 $a^b\equiv a^{b \mod \phi(m)+\phi(m)}\pmod m$
\item $a$ 为合数的情况
只证 $a$ 拆成两个素数的幂的情况,大于两个的用数学归纳法可证。
设 $a=a_1a_2,a_i=p_i^{k_i}$,$a_i$ 的循环长度为 $s_i$;
则 $s \mid lcm(s_1,s_2)$,由于 $s_1 \mid \phi(m),s_2 \mid \phi(m)$,那么 $lcm(s_1,s_2) \mid \phi(m)$,所以 $s \mid \phi(m)$, $r=\max(\lceil \frac{r_i}{k_i} \rceil) \le \max(r_i) \le \phi(m)$;
由 $r,s$ 与 $\phi(m)$ 的关系,依然可以得到 $a^b\equiv a^{b \mod \phi(m)+\phi(m)}\pmod m$;
证毕。
\end{enumerate}
