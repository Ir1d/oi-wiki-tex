
\subsection{大步小步算法}

\subsubsection{1.0 基础篇}

大步小步算法英文名:\textbf{baby-step gaint-step (BSGS)}.

该算法可以在 $O(\sqrt{q})$ 用于求解

$$
a^x \equiv b \bmod p
$$

其中 $p$ 是个质数的方程的解 $x$ 满足 $0 \le x < p$ .

令 $x = A \lceil \sqrt p \rceil - B$,其中 $0\le A,B \le \lceil \sqrt p \rceil$,

则有 $a^{A\lceil \sqrt p \rceil -B} \equiv b$,稍加变换,则有 $a^{A\lceil \sqrt p \rceil} \equiv ba^B$.

我们已知的是 $a,b$,所以我们可以先算出等式右边的 $ba^B$ 的所有取值,枚举 $B$,用 hash/map 存下来,然后逐一计算 $a^{A\lceil \sqrt p \rceil}$,枚举 $A$,寻找是否有与之相等的 $ba^B$,从而我们可以得到所有的 $x$,$x=A \lceil \sqrt p \rceil - B$.

注意到 $A,B$ 均小于 $\lceil \sqrt p \rceil$,所以时间复杂度为 $O(\sqrt q)$,用 map 的话会多一个 $\log$.

\href{http://www.lydsy.com/JudgeOnline/problem.php?id=2480}{BZOJ-2480} 是一道模板题(可能是权限题),\href{http://www.lydsy.com/JudgeOnline/problem.php?id=3122}{BZOJ-3122} 是一道略加变化的题,代码可以在 \href{https://blog.csdn.net/Steaunk/article/details/78988376}{Steaunk 的博客} 中看到.

\subsubsection{2.0 略微进阶篇}

求解

$$
x^a \equiv b \bmod p
$$

其中 $p$ 是个质数.

该模型可以通过一系列的转化为成\textbf{基础篇}中的模型,你可能需要一些关于  原根  的概念.

\textbf{原根的定义}为:对于任意数 $a$,满足 $(a,p)=1$,且 $t$ 为最小的\textbf{正整数}满足 $a^t \equiv 1 \bmod p$,则称 $t$ 是 $a$ 模 $p$ 意义下的次数,若 $t=\varphi(p)$,则称 $a$ 是 $p$ 的原根.

首先根据\textbf{原根存在的条件},对与所有的素数 $p>2$ 和正整数 $e$,当且仅当 $n=1,2,4,p^e,2p^e$ 时有原根,

那么由于式子中的模数 $p$ ,那么一定存在一个 $g$ 满足 $g$ 是 $p$ 的原根,即对于任意的数 $x$ 在模 $p$ 意义下一定有且仅有一个数 $i$,满足 $x = g^i$,且 $0 \le x,i < p$.

所以我们令 $x=g^c$,$g$ 是 $p$ 的原根(我们一定可以找到这个 $g$ 和 $c$),则为求 $(g^c)^a \equiv b \bmod p$ 的关于 $c$ 的解集,稍加变换,则有 $(g^a)^c \equiv b \bmod p$ ,于是就转换成了我们熟知的 \textbf{BSGS} 的基本模型了,即可在 $O(\sqrt p)$ 解决.

那么关键的问题就在于如何找到这个 $g$ 了?

关于对于存在原根的数 $p$ 有这样的\textbf{性质}:若 $t$ 是 $a$ 模 $p$ 的次数(这里蕴含了 $(a,p)=1$),那么对于任意的数 $d$,满足 $a^d \equiv 1 \bmod p$,则 $t \mid d$.

\textbf{PROOF}

记 $d = tq+r$,$0 \le r < t$.

$\because a^d \equiv a^{xq+r} \equiv (a^t)^qa^r \equiv a^r \equiv 1$.

$\because 0 \le r < t$,$t$ 是 $a$ 模 $p$ 的次数,即 $t$ 是最小的\textbf{正整数}满足 $a^t \equiv 1$.

$\therefore r = 0$.

即 $d = tq$,$t \mid d$

\textbf{Q.E.D.}

由此当 $p$ 是质数的时候还有这样的推论:如果不存在小于 $p$ 且整除 $p-1$ 正整数 $t$, 满足 $a^t \equiv 1$,那么又根据\textbf{费马小定理},有 $a^{p-1} \equiv 1$,所以 $p-1$ 是 $a$ 模 $p$ 的次数,即 $a$ 是 $p$ 的原根.

于是可以得到一种基于\textbf{原根分布}的算法来找原根,首先把 $p-1$ 的因数全部求出来,然后从 $2$ 到 $p-1$ 枚举,判断是否为原根,如果对于数 $g$,$\exists g^t \equiv 1 \bmod p$,$t$ 是 $p-1$ 的因数,则 $g$ 一定不是 $p$ 的原根.

看上去复杂度好像很爆炸(可能确实是爆炸的,但一般情况下,最小的原根不会很大).

\sout{基于一个\textbf{假设},原联系根是\textbf{均匀分布}的,我们\textbf{伪证明}一下总复杂度}:原根数量定理:数 $p$ 要么没有原根,要么有 $\varphi(\varphi(p))$ 个原根.

由于 $p$ 是质数,所以 $p$ 有 $\varphi(p-1)$ 个原根,所以大概最小的原根为 $\frac{p}{\varphi(p-1)}=O(\log\log n)$,由于求每一个数时要枚举一遍 $p-1$ 所有的因数 $O(\sqrt p)$ 来判断其是否为原根,最后再算上 \textbf{BSGS} 的复杂度 $O(\sqrt{p})$,则复杂度约为 $O(\sqrt{p}\log \log n)$.

\href{http://www.lydsy.com/JudgeOnline/problem.php?id=1319}{BZOJ-1319} 是一道模板题,代码可以在 \href{https://blog.csdn.net/Steaunk/article/details/78988376}{Steaunk 的博客} 中看到.

\subsubsection{3.0 扩展篇}

上文提到的情况是 $c$ 为素数的情况,如果 $c$ 不是素数呢?

这就需要用到扩展 BSGS 算法,不要求 $c$ 为素数!

扩展 BSGS 用到了同余的一条性质:

令 $d=gcd(a,c) ,a=m \times d,b=n \times d,p=k \times d$;

则 $m \times d \equiv b \times d \pmod {c \times d}$ 等价于 $m \equiv n \pmod k$

所以我们要先消除因子:

\begin{cppcode}
d = 1, num = 0, t = 0;
for (int t = gcd(a, c); t != 1; t = gcd(a, c)) {
  if (b % t) {
    \\无解
  }
  b /= t;
  c /= t;
  d *= a / t;
  num++;
}
\end{cppcode}

消除完后,就变成了 $d \times m^{x-num} \equiv n \pmod k$,令 $x=i \times m+j+num$,后面的做法就和普通 BSGS 一样了。

注意,因为 $i,j \le 0$,所以 $x \le num$,但不排除解小于等于 $num$ 的情况,所以在消因子之前做一下 $\Theta(\log_2 p)$ 枚举,直接验证 $a^i \mod c = b$,这样就能避免这种情况。
