
欧拉函数是什么? $\varphi(n)$ 表示的是小于等于 $n$ 和 $n$ 互质的数的个数。

比如说 $\varphi(1) = 1$。

当 n 是质数的时候,显然有 $\varphi(n) = n - 1$。

利用唯一分解定理,我们可以把一个整数唯一地分解为质数幂次的乘积,

设 $n = p_1^{k_1}p_2^{k_2} \cdots p_s^{k_s}$,其中 $p_i$ 是质数,那么定义 $\varphi(n) = n \times \prod_{i = 1}^s{\frac{p_i - 1}{p_i}}$

\subsubsection{欧拉函数的一些神奇性质}

\begin{itemize}
\item 欧拉函数是积性函数。
  积性是什么意思呢?如果有 $\gcd(a, b) = 1$,那么 $\varphi(a \times b) = \varphi(a) \times \varphi(b)$。
  特别地,当 $n$ 是奇数时 $\varphi(2n) = 2 \times \varphi(n)$。
\item $n = \sum_{d | n}{\varphi(d)}$
  利用  莫比乌斯反演  相关知识可以得出。
  也可以这样考虑:如果 $\gcd(k, n) = d$,那么 $\gcd(\frac{k}{d},\frac{n}{d}) = 1$。($k < n$)
  如果我们设 $f(x)$ 表示 $\gcd(k, n) = x$ 的数的个数,那么 $n = \sum_{i = 1}^n{f(x)}$。
  根据上面的证明,我们发现,$f(x) = \varphi(\frac{n}{x})$,从而 $n = \sum_{d | n}\varphi(\frac{n}{d})$。注意到约数 $d$ 和 $\frac{n}{d}$ 具有对称性,所以上式化为 $n = \sum_{d | n}\varphi(d)$。
\item 若 $n = p^k$,其中 $p$ 是质数,那么 $\varphi(n) = p^k - p^{k - 1}$。
  (根据定义可知)
\end{itemize}

\subsubsection{如何求欧拉函数值}

如果只要求一个数的欧拉函数值,那么直接根据定义质因数分解的同时求就好了。

\begin{cppcode}
int euler_phi(int n) {
  int m = int(sqrt(n + 0.5));
  int ans = n;
  for (int i = 2; i <= m; i++)
    if (n % i == 0) {
      ans = ans / i * (i - 1);
      while (n % i == 0) n /= i;
    }
  if (n > 1) ans = ans / n * (n - 1);
  return ans;
}
\end{cppcode}

如果是多个数的欧拉函数值,可以利用后面会提到的线性筛法来求得。

详见: 筛法求欧拉函数 
